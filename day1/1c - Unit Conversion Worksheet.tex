\documentclass[12pt, letterpaper]{article}
\usepackage{nth}
\usepackage{xcolor}
\usepackage{hyperref}
\usepackage[letterpaper,margin=1in]{geometry}
\usepackage{amssymb}
\usepackage{amsmath}
\usepackage[parfill]{parskip}
\usepackage{fullpage,bm}
\usepackage{tikz}
\usetikzlibrary{shapes, arrows, calc}

\hypersetup{colorlinks=false,linkbordercolor=red,linkcolor=green,pdfborderstyle={/S/U/W 1}}

\newcommand{\email}[1]{\href{mailto:#1}{#1}}
\newcommand{\U}[1]{\textrm{ #1}}
% \vspace{1\baselineskip}

\newcommand{\answergrid}[1]{
\hspace{-1cm}
\vspace{2em}
\tikz[remember picture, overlay] \node[inner sep=0, anchor=base] (tl) {}; %
\vspace{#1} \hfill%
\tikz[remember picture, overlay] \node[anchor=south west] at (tl.north) {\textbf{Answer:}};
\tikz[remember picture, overlay] \coordinate (br); %
\tikz[remember picture, overlay] \draw[step=5mm, blue!50]%
	let \p1=(tl.north), \p2=(br) in [yshift={mod(\y1-\y2, 5mm)}, xshift=\x1] %
	(0, 0) grid ( %
		{\x2 - \x1 - mod(\x2-\x1, 5mm) + 5mm + 0.1pt}, %
		{-#1 - mod(-#1, 5mm) - 0.1pt} %
	); %
}

\newcommand{\grid}[1]{
\hspace{-1cm}
\vspace{2em}
\tikz[remember picture, overlay] \node[inner sep=0, anchor=base] (tl) {}; %
\vspace{#1} \hfill%
\tikz[remember picture, overlay] \coordinate (br); %
\tikz[remember picture, overlay] \draw[step=5mm, blue!50]%
	let \p1=(tl.north), \p2=(br) in [yshift={mod(\y1-\y2, 5mm)}, xshift=\x1] %
	(0, 0) grid ( %
		{\x2 - \x1 - mod(\x2-\x1, 5mm) + 5mm + 0.1pt}, %
		{-#1 - mod(-#1, 5mm) - 0.1pt} %
	); %
}

\newcommand{\fillanswergrid}{
\hspace{-1cm}
\vspace{2em}
\tikz[remember picture, overlay] \node[inner sep=0, anchor=base] (tl) {}; %
\vfill\hfill%
\tikz[remember picture, overlay] \node[anchor=south west] at (tl.north) {\textbf{Answer:}};
\tikz[remember picture, overlay] \coordinate (br); %
\tikz[remember picture, overlay] \draw[step=5mm, blue!50]%
	let \p1=(tl.north), \p2=(br) in [yshift={mod(\y1-\y2, 5mm)}, xshift=\x1] %
	(0, 0) grid ( %
		{\x2 - \x1 - mod(\x2-\x1, 5mm) + 5mm + 0.1pt}, %
		{\y1 - mod(\y1-\y2, 5mm) + 0.1pt} %
	);
\clearpage
}

\newcommand{\fillgrid}{
\hspace{-1cm}
\tikz[remember picture, overlay] \node[inner sep=0, anchor=base] (tl) {}; %
\vfill\hfill%
\tikz[remember picture, overlay] \coordinate (br); %
\tikz[remember picture, overlay] \draw[step=5mm, blue!50]%
	let \p1=(tl.north), \p2=(br) in [yshift={mod(\y1-\y2, 5mm)}, xshift=\x1] %
	(0, 0) grid ( %
		{\x2 - \x1 - mod(\x2-\x1, 5mm) + 5mm + 0.1pt}, %
		{\y1 - mod(\y1-\y2, 5mm) + 0.1pt} %
	);
\clearpage
}


\newcommand{\header}[1]{
\begin{center}
{\bf
\huge #1

\large Summer STEM Academy: 

Measuring things which are very fast or very small

}
\end{center}

}

\renewcommand{\deg}{\ensuremath{^\circ}}


% \usepackage{nth}
% \usepackage{xcolor}
% \usepackage{hyperref}
% \usepackage[letterpaper,margin=1in]{geometry}
% \usepackage{amsmath}
% \usepackage[parfill]{parskip}
%
% \hypersetup{colorlinks=false,linkbordercolor=red,linkcolor=green,pdfborderstyle={/S/U/W 1}}
%
% \newcommand{\email}[1]{\href{mailto:#1}{#1}}
% \newcommand{\U}[1]{\textrm{ #1}}
% % \vspace{1\baselineskip}

\begin{document}
\header{Day 1 Worksheet: Converting Units}

In this workshop we will be measuring physical things like space and time. 
In order to do this, we need to understand a little bit about `units': what they are and how to convert between them.

By `units', I mean units of measurement, for example \textbf{distance} measurements like inches or kilometers, or \textbf{time} measurements like seconds or years.
 % e.g.~you are probably familiar with the metric (meters, kilograms) or imperial (feet, pounds) systems of measurements.
% For our purposes, there are two types we care about the most: distance and time.
You can think about these as physical things that exist in the universe, and we have different ways of representing them.
% For example: you might measure distance in microns, meters, feet, or light-years.
% Similarly you might measure time in nanoseconds, seconds, or years.
If you want, you can represent them anyway you want, but usually there is a best choice for every situation.
% In each case, it is usually more convenient to use one or the other representation, depending on the circumstances.
% Often, in science we need to understand how to convert between them: that's what we'll be doing right now.
Often, we need to be able to convert between them -- for example you might want to convert the speed of sound measurement you will make tomorrow from cm/ms (centimeters per millisecond) to m/s (meters per second) or mi/hr (miles per hour).

By the way: in the metric system, there is a convenient way to represent really big and really small numbers by modifying units (e.g. centimeters or kilometers instead of meters).
There is also `scientific notation', which is another convenient way to represent really big and really small numbers.
We'll talk more about both of these, and how they are related, tomorrow!

To get started, I found a nice YouTube video which explains basic unit conversion:\\ \href{https://www.youtube.com/watch?v=HRe1mire4Gc}{www.youtube.com/watch?v=HRe1mire4Gc}


By the way: for this class we will mostly care only about distance and time units.
Of course there are other types of units like energy, mass, power, etc., but we won't need to know anything about them.
If you're curious: feel free to ask!
The only other types of units we will use are Volts (V), which is a measure of electrical `potential energy'.
There's really only one way to represent them, unless you count milliVolts (1000 mV = 1V), but these are super easy to convert.
% (You may also see milliVolts -- 1000 mV = 1 V -- but that's pretty easy to convert.)

\clearpage


\section{Distance}
Now that we know how to convert between different types of units, all we really need to know is how all the different units are related.
Here's a table to help you out:


\begin{centering}
	
\begin{tabular}{r|l}
\textbf{Unit}&\textbf{Symbol / Relation}\\
\hline
inch & 1 in = 25.4 mm\\
foot & 1 ft = 12 in\\
yard & 1 yd = 3 ft\\
mile & 1 mi = 5,280 ft\\
meter & 1 m\\
centimeter & 100 cm = 1 m\\
millimeter & 1,000 mm = 1 m\\
kilometer & 1 km = 1,000 m
\end{tabular}

\end{centering}


\textbf{Exercises:}

\begin{enumerate}
\item How many mm are in 1 ft?

\answergrid{5cm}

\item How many m are in 1 mi?

\answergrid{5cm}
\end{enumerate}

\clearpage

\section{Time}
Time is even simpler, since we don't have to worry about conversion between metric and imperial units!

\begin{centering}
	
\begin{tabular}{r|l}
\textbf{Unit}&\textbf{Symbol / Relation}\\
\hline
nanosecond & 1,000,000,000 ns = 1 s\\
microsecond & 1,000,000 $\mu$s = 1 s\\
millisecond & 1,000 ms = 1 s\\
second & 1 s\\
minute & 1 min = 60 s\\
hour & 1 hr = 60 min\\
day & 1 day = 24 hr\\
year & 1 year = 365.25 days
\end{tabular}

\end{centering}

% Note that nanoseconds/microseconds are easier to deal with in terms of scientific notation -- we'll discuss that tomorrow!

\textbf{Exercises:}

\begin{enumerate}
\item How many seconds are in a day?

\answergrid{5cm}

\item How may minutes are in a year?

\answergrid{5cm}

\item How may $\mu$s are in 1 ms?

\answergrid{5cm}
\end{enumerate}


\section{Combinations of Units}
We can also combine these to produce things like area (distance \emph{times} distance) or speed (distance \emph{divided by} time).
Conversion is relatively straightforward, and is probably best illustrated by example.

\textbf{Example:}

The record for the 100 yard dash in 9.07 s. 
What is this in terms of km/hr?

\begin{centering}
	
\begin{align}
	100 \U{yd} &= 300 \U{ft} \times \frac{1 \U{m}}{3.28084 \U{ft}}\\
	&= 91.44 \U{m}\\
	&= 0.09144 \U{km}\\
	9.07 \U{s} &= 9.07 \U{s} \times \frac{1 \U{hr}}{3600 \U{s}}\\
	&= 0.002519 \U{hr}\\
	\Rightarrow \frac{100 \U{yd}}{9.07 \U{s}} &= \frac{0.09144 \U{km}}{0.002519 \U{hr}}\\
	&= 36.29 \U{km/hr}
\end{align}

\end{centering}

\clearpage

\textbf{Exercises:}

\begin{enumerate}

\item What is 1 m/s in mi/hr?

\answergrid{5cm}
\item What is 50 mi/hr in cm/ms?

\answergrid{5cm}
\item The earth orbits the sun at a speed of about 67,000 mi/hr.  What is this m/ms?

\answergrid{5cm}
\item (continued) What about this speed in mm/$\mu$s?

\answergrid{5cm}
\item (continued) What about this speed in $\mu$m/ns?

\answergrid{5cm}
\item For some reason, in the US we measure land area in terms of `acres', which is one chain (66 ft) by one furlong (660 ft).
What is this in terms of m$^2$ (meters squared)?

\answergrid{5cm}
\item What is 1 mi$^2$ (miles squared) in terms of m$^2$?

\answergrid{5cm}
\item How many acres are in 1 km$^2$?

\fillanswergrid
% \item From the data you took at the end of the oscilloscope lab, you should be able to estimate the speed of sound.
% Do this using just two of your data points, and write your answer in terms of both mi/hr and m/s.
% (Ask if you need help: you'll need to make some logical leaps!)
%
% \fillanswergrid
%
% \fillgrid

\end{enumerate}

% \section{Other Units}
%
% \subsection{Electrical Voltage}
% We'll also encounter `voltage' in this class, which is a measure of electrical energy (or more precisely: energy per unit of charge).
% This is always measured in Volts (V): there is no metric or imperial alternative!
% You may also encounter millivolts (mV) which is 1/1000 V.
% You won't need to know anything else about volts, and you won't need to convert between them!

% \subsection{Bonus: Mass and Energy}
% \textbf{This is not important to complete, but if you have extra time give it a try!}
%
% There are many other types of units, but perhaps the most important are:
% \begin{itemize}
% 	\item mass: usually measured in grams (g) or kilograms (kg).  1 kg is approximately 2.2 lbs.
% 	\item energy: measured in Joules (J).
% 	\item power: power is energy per time, measured in Watts (W).  1 W = 1 J/s.
% \end{itemize}
%
%
% \textbf{Exercises:}
% (Note: these are harder than the last ones, and you may need some help to figure them out!)
%
% \begin{enumerate}
% \item When you get a power bill, the total energy you used is listed in terms of kilowatt hours (kWh = 1 kW $\times$ 1 hr).  How many Joules is 1 kWh?
% (Note: 1 kW = 1000 W, and 1 J = 1 W $\times$ 1 s)
%
% \fillanswergrid
%
% \item The energy required to accelerate an object is given by $E = \frac{1}{2} m v^2$, where $E$ is measured in J, $m$ is measured in kg and $v$ is measured in m/s.
% So... how many kWh is needed to accelerate a Tesla model 3 (which weighs about 4000 lbs) to 75 mph?
%
% \answergrid{7cm}
% \item The energy required to lift an object in gravity is given by $E = mgh$, where $g = 9.8$ m/s is the acceleration of gravity and $h$ is measured in m.
% So... how many kWh is needed to lift a Tesla model 3 from sea level to Tioga pass in Yosemite ($h$ = 9,943 ft)?
% Approximately what percentage of the battery in the car is this?  (You'll have to look up the capacity of a Tesla battery!)
%
% \fillanswergrid
%
% \item How many kWh does it take to heat a gallon of water from 20 C to boiling?
% (To figure this out, you need to know that it takes 4.2 J to heat 1 g of water by 1 C... there are also some other conversions you'll need to figure out on your own!)
%
% \fillanswergrid
% \end{enumerate}

\clearpage
\fillgrid

\end{document}