\documentclass[12pt, letterpaper]{article}
\usepackage{nth}
\usepackage{xcolor}
\usepackage{hyperref}
\usepackage[letterpaper,margin=1in]{geometry}
\usepackage{amssymb}
\usepackage{amsmath}
\usepackage[parfill]{parskip}
\usepackage{fullpage,bm}
\usepackage{tikz}
\usetikzlibrary{shapes, arrows, calc}

\hypersetup{colorlinks=false,linkbordercolor=red,linkcolor=green,pdfborderstyle={/S/U/W 1}}

\newcommand{\email}[1]{\href{mailto:#1}{#1}}
\newcommand{\U}[1]{\textrm{ #1}}
% \vspace{1\baselineskip}

\newcommand{\answergrid}[1]{
\hspace{-1cm}
\vspace{2em}
\tikz[remember picture, overlay] \node[inner sep=0, anchor=base] (tl) {}; %
\vspace{#1} \hfill%
\tikz[remember picture, overlay] \node[anchor=south west] at (tl.north) {\textbf{Answer:}};
\tikz[remember picture, overlay] \coordinate (br); %
\tikz[remember picture, overlay] \draw[step=5mm, blue!50]%
	let \p1=(tl.north), \p2=(br) in [yshift={mod(\y1-\y2, 5mm)}, xshift=\x1] %
	(0, 0) grid ( %
		{\x2 - \x1 - mod(\x2-\x1, 5mm) + 5mm + 0.1pt}, %
		{-#1 - mod(-#1, 5mm) - 0.1pt} %
	); %
}

\newcommand{\grid}[1]{
\hspace{-1cm}
\vspace{2em}
\tikz[remember picture, overlay] \node[inner sep=0, anchor=base] (tl) {}; %
\vspace{#1} \hfill%
\tikz[remember picture, overlay] \coordinate (br); %
\tikz[remember picture, overlay] \draw[step=5mm, blue!50]%
	let \p1=(tl.north), \p2=(br) in [yshift={mod(\y1-\y2, 5mm)}, xshift=\x1] %
	(0, 0) grid ( %
		{\x2 - \x1 - mod(\x2-\x1, 5mm) + 5mm + 0.1pt}, %
		{-#1 - mod(-#1, 5mm) - 0.1pt} %
	); %
}

\newcommand{\fillanswergrid}{
\hspace{-1cm}
\vspace{2em}
\tikz[remember picture, overlay] \node[inner sep=0, anchor=base] (tl) {}; %
\vfill\hfill%
\tikz[remember picture, overlay] \node[anchor=south west] at (tl.north) {\textbf{Answer:}};
\tikz[remember picture, overlay] \coordinate (br); %
\tikz[remember picture, overlay] \draw[step=5mm, blue!50]%
	let \p1=(tl.north), \p2=(br) in [yshift={mod(\y1-\y2, 5mm)}, xshift=\x1] %
	(0, 0) grid ( %
		{\x2 - \x1 - mod(\x2-\x1, 5mm) + 5mm + 0.1pt}, %
		{\y1 - mod(\y1-\y2, 5mm) + 0.1pt} %
	);
\clearpage
}

\newcommand{\fillgrid}{
\hspace{-1cm}
\tikz[remember picture, overlay] \node[inner sep=0, anchor=base] (tl) {}; %
\vfill\hfill%
\tikz[remember picture, overlay] \coordinate (br); %
\tikz[remember picture, overlay] \draw[step=5mm, blue!50]%
	let \p1=(tl.north), \p2=(br) in [yshift={mod(\y1-\y2, 5mm)}, xshift=\x1] %
	(0, 0) grid ( %
		{\x2 - \x1 - mod(\x2-\x1, 5mm) + 5mm + 0.1pt}, %
		{\y1 - mod(\y1-\y2, 5mm) + 0.1pt} %
	);
\clearpage
}


\newcommand{\header}[1]{
\begin{center}
{\bf
\huge #1

\large Summer STEM Academy: 

Measuring things which are very fast or very small

}
\end{center}

}

\renewcommand{\deg}{\ensuremath{^\circ}}


\begin{document}
\header{Day 2: Speed of Sound}
\normalsize

\section{Part 1}

Yesterday you should have finished the day connecting the ultrasound receiver / transmitter.
If you didn't get that far, go back to the last part of the day 1 worksheet and start at the `Ultrasound' section.

\begin{enumerate}
	\item If you move the transmitter near the receiver, you should see that you can transmit a signal!  Try attaching both to the metal rail, and move them around.  What happens to the signal?

	\answergrid{3cm}
	\item With both screwed down, switch the black and red alligator clips on the transmitter.  What happens to the signal on the receiver?

	\answergrid{3cm}
	\item Now try changing the frequency of the transmitter -- what happens to the signal on the receiver?  How high or low can you go and still get a decent signal?  Which is the `best' frequency to use?

	\answergrid{3cm}
\end{enumerate}

Since the transmitter is sending sound to the receiver, we should get a delay based on how long it takes sound to move between them.
We can use this to measure the speed of sound!

To do so, we don't want a continuous signal, but rather a `burst'.
We can set this up with the `burst' button on the function generator.
In the burst menu set up the following options on the first page: \textbf{Period:} 10 ms, \textbf{NCycle}, \textbf{Source:} internal.
On the second page: \textbf{Cycles:} 10.

You should be able to see that the signal on the oscilloscope only oscillates for 10 periods at a time now.
If it doesn't work ask for help (the burst menu is super confusing).

Set the receiver and transmitter about 30 cm apart -- you should now be able to adjust the oscilloscope settings to see both the burst and the \textbf{delayed response} of the receiver.
If you move the receiver relative to the transmitter you should see the receiver signal shift.

\begin{enumerate}
	\item To measure delay caused by sound we need to measure the time interval between the transmitter signal and the receiver response.
	But... both signals have some width in time, so how would we determine where to start and stop our `timer'?
	Sketch a graph of the signals and mark what you will use as the reference points.
	Ask the professor or assistants to look at your answer before moving on!

	\fillanswergrid
	\item Measure the delay (using the method you described above) for a bunch of different separations between the transmitter and receiver.
	Make a chart with at 5--10 values of distance (separation between receiver and transmitter) and measured delay.

	\fillanswergrid
\end{enumerate}


% \section{If you finish early...}
% \begin{itemize}
% 	\item Try changing the number of `cycles' in the burst, and see how this affects the signal.  Could you make a better measurement of the delay?  (And if so, why?)
% 	\item Put the function generator in continuous (i.e.~not burst) mode and then look at the transmitted/received signals in X-Y mode.  Discuss if you could use this to measure delays and/or the speed of sound.
% 	\item Spend some time exploring the different functions of the oscilloscope and function generator.  You can make some cool plots with X-Y mode and signals of different frequencies.
% \end{itemize}

\section{Part 2}

We're going to make a spreadsheet to hold the data you just took, and fit a line to it.
To get started, let's watch a YouTube tutorial on GoogleDocs:\\ \href{https://www.youtube.com/watch?v=fHxfoQsc8hc}{www.youtube.com/watch?v=fHxfoQsc8hc}

One thing this video doesn't mention: you can include your column headings in the spreadsheet, and if you do so your axes labels will be automatically included.
In your case, cell A1 should be `delay (ms)', and B1 should be `distance (cm)'.
The data then goes below these headings, and you select the whole thing (include headings) when you make a graph.

\begin{enumerate}
	\item Make a GoogleDocs spreadsheet and fill in your data.
	\item Make a plot of this data and fit this using a trendline.  
	(Note: to get a good fit, you may need to exclude some of the points: consult with the professor or assistants on your data before moving on!)
	\item Using your fit, you should now be able to measure the speed of sound!  What is it in therms of: (a) cm/ms, (b) m/s, and (c) mi/hr?
	
	\answergrid{5cm}
\end{enumerate}

\section{Part 3}
Rather than use a `burst' signal, we can also use a continuous signal and measure the wavelength!
As the professor mentioned (briefly) on the first day, if you have the period ($T$) or frequency ($f$) and wavelength ($\lambda$) of a moving wave, you can compute the velocity of the wave with:
\begin{align}
	v &= \frac{\lambda}{T} = \lambda f
\end{align}
Discuss as a group how to measure the wavelength using the `X-Y' mode.
Then try it!

If you finish early... ask the professor or assistants about using the `wavefit' program on the laptops to improve your phase measurements, and get a more accurate result for the wavelength.

\fillgrid

\clearpage

\fillgrid

\end{document}


