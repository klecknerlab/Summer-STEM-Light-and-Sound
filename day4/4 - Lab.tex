\documentclass[12pt, letterpaper]{article}
\usepackage{nth}
\usepackage{xcolor}
\usepackage{hyperref}
\usepackage[letterpaper,margin=1in]{geometry}
\usepackage{amssymb}
\usepackage{amsmath}
\usepackage[parfill]{parskip}
\usepackage{fullpage,bm}
\usepackage{tikz}
\usetikzlibrary{shapes, arrows, calc}

\hypersetup{colorlinks=false,linkbordercolor=red,linkcolor=green,pdfborderstyle={/S/U/W 1}}

\newcommand{\email}[1]{\href{mailto:#1}{#1}}
\newcommand{\U}[1]{\textrm{ #1}}
% \vspace{1\baselineskip}

\newcommand{\answergrid}[1]{
\hspace{-1cm}
\vspace{2em}
\tikz[remember picture, overlay] \node[inner sep=0, anchor=base] (tl) {}; %
\vspace{#1} \hfill%
\tikz[remember picture, overlay] \node[anchor=south west] at (tl.north) {\textbf{Answer:}};
\tikz[remember picture, overlay] \coordinate (br); %
\tikz[remember picture, overlay] \draw[step=5mm, blue!50]%
	let \p1=(tl.north), \p2=(br) in [yshift={mod(\y1-\y2, 5mm)}, xshift=\x1] %
	(0, 0) grid ( %
		{\x2 - \x1 - mod(\x2-\x1, 5mm) + 5mm + 0.1pt}, %
		{-#1 - mod(-#1, 5mm) - 0.1pt} %
	); %
}

\newcommand{\grid}[1]{
\hspace{-1cm}
\vspace{2em}
\tikz[remember picture, overlay] \node[inner sep=0, anchor=base] (tl) {}; %
\vspace{#1} \hfill%
\tikz[remember picture, overlay] \coordinate (br); %
\tikz[remember picture, overlay] \draw[step=5mm, blue!50]%
	let \p1=(tl.north), \p2=(br) in [yshift={mod(\y1-\y2, 5mm)}, xshift=\x1] %
	(0, 0) grid ( %
		{\x2 - \x1 - mod(\x2-\x1, 5mm) + 5mm + 0.1pt}, %
		{-#1 - mod(-#1, 5mm) - 0.1pt} %
	); %
}

\newcommand{\fillanswergrid}{
\hspace{-1cm}
\vspace{2em}
\tikz[remember picture, overlay] \node[inner sep=0, anchor=base] (tl) {}; %
\vfill\hfill%
\tikz[remember picture, overlay] \node[anchor=south west] at (tl.north) {\textbf{Answer:}};
\tikz[remember picture, overlay] \coordinate (br); %
\tikz[remember picture, overlay] \draw[step=5mm, blue!50]%
	let \p1=(tl.north), \p2=(br) in [yshift={mod(\y1-\y2, 5mm)}, xshift=\x1] %
	(0, 0) grid ( %
		{\x2 - \x1 - mod(\x2-\x1, 5mm) + 5mm + 0.1pt}, %
		{\y1 - mod(\y1-\y2, 5mm) + 0.1pt} %
	);
\clearpage
}

\newcommand{\fillgrid}{
\hspace{-1cm}
\tikz[remember picture, overlay] \node[inner sep=0, anchor=base] (tl) {}; %
\vfill\hfill%
\tikz[remember picture, overlay] \coordinate (br); %
\tikz[remember picture, overlay] \draw[step=5mm, blue!50]%
	let \p1=(tl.north), \p2=(br) in [yshift={mod(\y1-\y2, 5mm)}, xshift=\x1] %
	(0, 0) grid ( %
		{\x2 - \x1 - mod(\x2-\x1, 5mm) + 5mm + 0.1pt}, %
		{\y1 - mod(\y1-\y2, 5mm) + 0.1pt} %
	);
\clearpage
}


\newcommand{\header}[1]{
\begin{center}
{\bf
\huge #1

\large Summer STEM Academy: 

Measuring things which are very fast or very small

}
\end{center}

}

\renewcommand{\deg}{\ensuremath{^\circ}}


% \usepackage{nth}
% \usepackage{xcolor}
% \usepackage{hyperref}
% \usepackage[letterpaper,margin=1in]{geometry}
% \usepackage{amsmath}
% \usepackage[parfill]{parskip}
%
% \hypersetup{colorlinks=false,linkbordercolor=red,linkcolor=green,pdfborderstyle={/S/U/W 1}}
%
% \newcommand{\email}[1]{\href{mailto:#1}{#1}}
% \newcommand{\U}[1]{\textrm{ #1}}
% % \vspace{1\baselineskip}

\begin{document}
\header{Day 4 Worksheet: Measuring the Speed of Light}
Our goal today is to measure the speed of light as \emph{accurately and precisely} as possible.
If you don't know what this means, we'll talk about it in class or you can look at wikipedia: \href{https://en.wikipedia.org/wiki/Accuracy_and_precision}{en.wikipedia.org/wiki/Accuracy\_and\_precision}

\textbf{We're going to make this a competition: we will see who can get the best result, and we will give awards based on your score!}

We will do this in two phases:
\begin{enumerate}
	\item First you will build an apparatus to measure the speed of light, and come up with a written plan for making your measurement.
	You should try testing it out, and modifying it to improve your accuracy.
	\item Once you have finalized a plan, you will execute your plan 4--10 different times.
	At the end, you have at least 4 different independent measurements of the speed of light.
\end{enumerate}

The design of the experiment is up to you, and you are welcome to get creative!
You should consider:
\begin{itemize}
	\item Where will you place the laser, photodiode, and mirror (assuming you are using a mirror)?
	\item How far does the laser need to travel to get an accurate measurement?  How long can you go?
	  You aren't restricted to the black T-slot rail, however, your experiment must fit in the room!
	\item How will you measure the distance the laser travels?
	\item What resistor will you use for the photodiode?
	\item What frequency will you use to modulate the laser?
	\item How will you correct for `offset' which comes from the electronics?
	\item How many individual data points go into each measurement?  1, 2, or many?  Do you want to fit a line?
	\item Will making more measurements improve your precision or accuracy?
	\item Do you need anything that isn't already in the lab?  If so, ask, and maybe we can borrow something from my lab.
\end{itemize}

You will receive two scores:
\begin{enumerate}
	\item `Precision': how repeatable is your measurement?  This will be computed from the \textbf{standard deviation} of your 4--10 measurements.  Your score will be based on the relative precision compared to the accepted value (see equation/table below).
	% For example: if your standard deviation between measurements was $3 \times 10^6$ m/s, this is 1\% of the actual value (about $3 \times 10^8$ m/s. Thus you would receive 20 points (see below).
	\item `Accuracy': how close is your average value to the exact value (299,792,458 m/s)?  The difference between your \textbf{average} value and the accepted one is used to obtain a score below.  Once restriction: your accuracy score is \emph{at most} your precision score + 10 points.  (This is to prevent a group from getting lucky on their average value, even if their precision is terrible!)
\end{enumerate}

The score for each is be computed as:
 % (error is the standard deviation \emph{or} difference from the accepted value, depending on which score we are computing)

\begin{centering}

\begin{align}
	\textrm{score} = 10 \log_{10} \left(\frac{\textrm{exact value}}{\textrm{error}}\right)
\end{align}

\begin{tabular}{r|l}
	\textbf{Error}&\textbf{Points}\\
	\hline
	10\% & 10\\
	3.2\% & 15\\
	1\% & 20\\
	0.32\% & 25\\
	0.1\% & 30\\
	0.032\% & 35\\
	0.01\% & 40
\end{tabular}

\end{centering}

Additional rules:
\begin{enumerate}
	\item Once you start executing your plan, you must stick to it.  You can (and should) do preliminary measurements to test your technique before you start your final runs.
	\item You are welcome to try different photodiodes, lasers, resistors, etc.~if you think it will make a difference.
	\item You must execute your plan at least 4 times.  If you do it more than 4 times, all the values will be used for your scores -- you can not `drop' bad measurements.
	\item A `measurement' can use as many data points as you would like.  For example, perhaps you measured the delay at 5 different distances at fit a line to this -- in this case you would take 20 total data points if you make 4 `measurements'.
\end{enumerate}


\textbf{Example:}
 
Group 1 measured the speed of light 4 times and obtained: (2.941, 2.993, 3.051, 3.002) $\times 10^8$ m/s.
The standard deviation of this measurement is 0.0451 $\times 10^8$ m/s, which is a 1.5\% error for 18.2 points in the precision category.
The average is 2.9968 $\times 10^8$ m/s, which is a 0.039\% error for 34.1 points in the accuracy category.
However, the accuracy score is higher than the precision score + 10, which is not allowed, so it is capped at 28.2 points -- this results in a total score of 52.3 points.


Group 2 measured the speed of light 5 times and obtained: (2.801, 2.805, 2.804, 2.806, 2.802) $\times 10^8$ m/s.
The standard deviation of this measurement is 0.002 $\times 10^8$ m/s, which is a 0.07\% error for 31.6 points in the precision category.
The average is 2.803 $\times 10^8$ m/s, which is a 6.48\% error for 11.9 points in the accuracy category.
This results in a total score of 43.5 points.  (The cap on accuracy does not apply here.)

Notice that although group 2 had very precise measurements -- they got almost exactly the same result each time, their \emph{average} was very far off.  
This suggests they had a systematic error: perhaps they didn't correct for offset properly or had some other problem -- you should look out for these types of errors when you design your experiment!
Group 1 had a larger deviation between their measurements (and hence a lower precision score), but their average was very close to the true value.
In this case, Group 1 would win!

\vspace{2em}

{\Large Your measurement plan:}

\fillgrid
\fillgrid

\end{document}